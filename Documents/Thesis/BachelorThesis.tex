\documentclass[a4paper, 12pt]{article}

%•	Lucrarea se recomandă a avea între 30 și 60 de pagini
%•	Format: A4, fonturi de 12pt, distanță de 1.5 între rânduri
%•	Obligatoriu paginile sunt numerotate
%•	Capitolele vor începe pe pagină nouă
% margini: top=2.5cm,left=3cm,right=2.5cm,bottom=2.5cm
\usepackage{comment}
\usepackage{setspace}

% for bibliography
\usepackage[english]{babel}
\usepackage[nottoc]{tocbibind}
% for clicking on a cite leading to bibliography
\usepackage[colorlinks=true,linkcolor=black,citecolor=blue,urlcolor=blue]{hyperref}
% specificatii in legatura cu margini
\usepackage[top=2.5cm,left=3cm,right=2.5cm,bottom=2.5cm]{geometry}


% Redefine \maketitle to customize title layout
\makeatletter
\renewcommand{\maketitle}{
    \begin{center}
            \normalsize{BABEŞ-BOLYAI UNIVERSITY CLUJ-NAPOCA}\par % University name
            \normalsize{FACULTY OF MATHEMATICS AND COMPUTER SCIENCE}\par % Faculty name
            \normalsize{COMPUTER SCIENCE IN ROMANIAN SPECIALIZATION}\par
        \vspace{21em} % Vertical space

        {\LARGE\@title\par} % Title in large font
        \vspace{21em} % Vertical space

        \textbf{Supervisor}\hspace{20em}\textbf{Author}\par
        Prof.dr. Horia F. Pop\hspace{16em}{\large\@author\par} % Author
        \vspace{3em} % Vertical space

        {\large\@date\par} % Date
    \end{center}
}
\makeatother

\title{
    DIPLOMA THESIS \\
    Alzheimer's Disease Detection
}
\author{Ichim Ștefan}
\date{2024}

% Set line spacing to 1.5
\onehalfspacing

\begin{document}

% title page
\maketitle
\newpage

% ------------------------------------ABSTRACT--------------------------------------
% abstract page
\begin{abstract}
    Lorem ipsum Lorem ipsum Lorem ipsum Lorem ipsum
    Lorem ipsum Lorem ipsum Lorem ipsum Lorem ipsum
    Lorem ipsum Lorem ipsum Lorem ipsum Lorem ipsum
    Lorem ipsum Lorem ipsum Lorem ipsum Lorem ipsum
    Lorem ipsum Lorem ipsum Lorem ipsum Lorem ipsum
    Lorem ipsum Lorem ipsum Lorem ipsum Lorem ipsum
    Lorem ipsum Lorem ipsum Lorem ipsum Lorem ipsum
    Lorem ipsum Lorem ipsum Lorem ipsum Lorem ipsum
\end{abstract}
\newpage

% ------------------------------------CONTENTS--------------------------------------
\tableofcontents
\newpage

% ----------------------------------INTRODUCTION------------------------------------
\section{Introduction}
There is no denying that humanity stands at a previously inconceivable point in healthcare and medicine,
which naturally have led to hindrances in senescence, populations increasingly reaching older stages of life.
Furthermore, studies which take into account multiple case scenarios show that population is expected to reach
9.2 billion by the age of 2050, leading to an uprise of 21\% in the elderly. \cite{KC2017181}

With that being said, researchers' concern has has taken a turn towards diseases occurring at these later
parts of human lives, some of them considered treatable while others less so. One of such disorders is
Alzheimer's Disease, or AD, considered to be the most likely predecessor of dementia. Alzheimer's Disease
is a brain disease, neurodegenerative, which in time diminishes cognitive skills such as memory, thinking
and speaking, and in due course even removes the ability of accomplishing simple activities.
On top of that, it is an incurable disorder, which only underlines even further the reasons why early
detection stand of such great importance, so that necessary

\newpage


\bibliographystyle{plainnat} % Specify bibliography style
\bibliography{references}

\end{document}
