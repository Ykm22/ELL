\documentclass[a4paper, 12pt]{article}

%•	Lucrarea se recomandă a avea între 30 și 60 de pagini
%•	Format: A4, fonturi de 12pt, distanță de 1.5 între rânduri
%•	Obligatoriu paginile sunt numerotate
%•	Capitolele vor începe pe pagină nouă
% margini: top=2.5cm,left=3cm,right=2.5cm,bottom=2.5cm
\usepackage{comment}
\usepackage{setspace}

% for bibliography
\usepackage[english]{babel}
\usepackage[nottoc]{tocbibind}
% for clicking on a cite leading to bibliography
\usepackage{hyperref}
\hypersetup{
    colorlinks=true,
    linkcolor=black,
    citecolor=blue,
    urlcolor=blue,
    filecolor=blue, 
}
% specificatii in legatura cu margini
\usepackage[top=2.5cm,left=3cm,right=2.5cm,bottom=2.5cm]{geometry}

\usepackage{graphicx}

% Redefine \maketitle to customize title layout
\makeatletter
\renewcommand{\maketitle}{
    \begin{center}
            \normalsize{BABEŞ-BOLYAI UNIVERSITY CLUJ-NAPOCA}\par % University name
            \normalsize{FACULTY OF MATHEMATICS AND COMPUTER SCIENCE}\par % Faculty name
            \normalsize{COMPUTER SCIENCE IN ROMANIAN SPECIALIZATION}\par
        \vspace{21em} % Vertical space

        {\LARGE\@title\par} % Title in large font
        \vspace{21em} % Vertical space

        \textbf{Supervisor}\hspace{20em}\textbf{Author}\par
        Prof.dr. Horia F. Pop\hspace{16em}{\large\@author\par} % Author
        \vspace{3em} % Vertical space

        {\large\@date\par} % Date
    \end{center}
}
\makeatother

\title{
    DIPLOMA THESIS \\
    Alzheimer's Disease Detection
}
\author{Ichim Ștefan}
\date{2024}

% Set line spacing to 1.5
\onehalfspacing

\begin{document}

% title page
\maketitle
\newpage

% ------------------------------------ABSTRACT--------------------------------------
% abstract page
\begin{abstract}
    Lorem ipsum Lorem ipsum Lorem ipsum Lorem ipsum
    Lorem ipsum Lorem ipsum Lorem ipsum Lorem ipsum
    Lorem ipsum Lorem ipsum Lorem ipsum Lorem ipsum
    Lorem ipsum Lorem ipsum Lorem ipsum Lorem ipsum
    Lorem ipsum Lorem ipsum Lorem ipsum Lorem ipsum
    Lorem ipsum Lorem ipsum Lorem ipsum Lorem ipsum
    Lorem ipsum Lorem ipsum Lorem ipsum Lorem ipsum
    Lorem ipsum Lorem ipsum Lorem ipsum Lorem ipsum
\end{abstract}
\newpage

% ------------------------------------CONTENTS--------------------------------------
\tableofcontents
\newpage

% ----------------------------------INTRODUCTION------------------------------------
\section{Introduction}
There is no denying that humanity stands at a previously inconceivable point in healthcare and medicine,
which naturally have led to hindrances in senescence, populations increasingly reaching older stages of life.
Furthermore, studies which take into account multiple case scenarios show that population is expected to reach
9.2 billion by the age of 2050, leading to an uprise of 21\% in the elderly. \cite{KC2017181}

With that being said, researchers' concern has has taken a turn towards diseases occurring at these later
parts of human lives, some of them considered treatable while others less so.
One of such disorders is Alzheimer's Disease, or AD, considered to be the most likely predecessor of dementia.
Alzheimer's Disease is a brain disease, neurodegenerative, which in time diminishes cognitive skills such as memory,
thinking and speaking, and in due course even removes the ability of accomplishing simple activities vital to one's
daily life.
On top of that, it is an incurable disorder, which only underlines even further the reasons why early
detection stand of such great importance, so that appropriate actions can be taken by both the medical team
and the one diagnosed, along with their relatives and close ones.

\subsection{Disease Summary}

The brain of a healthy human represents a cluster of neurons by the number of bilions which together amount to
what actions and reactions we have, through a process of signal propagating.
Through our sensory mechanism, which includes hearing and seeing, receptors carry out the tasks of sending signals
(Fig \ref{fig:neuron-communication}) using designated channels all the way to the neurons inside the brain, where new
specific signals are formed and sent back, resulting in what we call actions. \cite{Sivadas2020HowDM}

\begin{figure}[htbp]
    \centering
    \includegraphics[width=0.65\textwidth]{figs/neuron-communication.jpg}
    \caption{Communication between neurons}
    \label{fig:neuron-communication}
\end{figure}

Alzheimer's Disease intervenes in this process by gradually decreasing the utility function of each neuron, leading to
the atrophy of the brain's proficiencies, as neurons imminently die one by one.

There are three major factors included in the dynamic between AD and neurons.
First of all, a key advantage of neurons which many other cells lack, and which accomplishes
their long survival, is the ability to repair themselves, form new connections, or changing current ones' magnitude.
Secondly, synaptic connections, which solidify the signal transmission process, and lastly the intake of glucose and
oxygen necessary for their normal functioning.
It is believed these fundamental attributes of a healthy human receive considerable drawbacks upon the disease's presence.
\cite{NIH1}

\subsection{Causes} %* -- Introduction/Causes
While the factors which lead to Alzheimer's Disease are not yet properly understood, past research and studies prove that some
of the most commonly met criterias which lead to a diagnostic include genetic inheritance - chances of developing Alzheimer's
Disease increase by 30\% when another close relative suffers from it \cite{HMS20192801}, lifestyle and environmental factors.

\subsubsection*{Genetical Inheritance} %* -- Introduction/Causes/GeneticalInheritance
Genes represent instructions passed down from generation to generation, which contain information regarding how various cells
need to behave. Some roles played by these include defining one's height, or the color of hair and eyes.

Advances in genetic research have led to discover 80 genetic areas that can possibly play a part in AD development \cite{NIH2}.
One of the more known genes which raises the risk of Alzheimer's Disease is the apolipoprotein E (APOE) gene, which comes in forms such as
$\epsilon_2, \epsilon_3, \epsilon_4$. A pair of two such APOE genes, one from each parent, gets passed down to the next generation
resulting in 6 possible cases. Among them, the $\left(\epsilon_4,\epsilon_4\right)$ combination having the highest risk of AD,
only increasing, not guaranteeing it, and in contrast, $\epsilon_2$ provides a higher degree of protection against it.

\subsubsection*{External Factors} %* -- Introduction/Causes/ExternalFactors
Besides genetical inheritance, researchers have drawn conclusions regarding causes of Alzheimer's Disease to contain a plethora
of other outside factors, which we can have a higher influence on.
Among these can be found vascular conditions - high blood pressure, heart diseases - and metabolic diseases - obesity and diabetes
\cite{NIH2}.

% todo - 15/04 -> symptoms, process of diagnosios, how scans work
\subsection{Symptoms} %* -- Introduction/Symptoms
Before beginning the discussion about its effects, a noteworthy fact is that brain structure modifications, whether they may be
neurofibrillary tangles or plaques of amyloid, occur several years before any cognitive issues manifest at all, a stage of the disease titled
preclinical. With that being said, their presence does not inevitably lead to dementia.

Besides preclinical stage, AD has been classified into three others: mild, moderate, severe.

\subsubsection*{Early-stage (Mild)} %* -- Introduction/Symptoms/Early-stage
A person which suffers from early-stage Alzheimer's Disease can still function normally on their own, without mandatory outside benefactors.
However, changes appear in memory skills, starting to forget recently gained information, such as names at social gatherings, objects placements
and losing the reasoninng behind starting certain activities.

It is important to understand these memory setbacks are hardly noticeable by the affected one, more commonly than not leaving it up to their
surrounding group of people and friends to pinpoint them and initiate medical visits.

\subsubsection*{Middle-stage (Moderate)} %* -- Introduction/Symptoms/Middle-stage
Here, over the course of many years, cognitive skills start degrading, with the diagnosed person needing increasing help from other people.
Previous rare memory losses become the norm, and even more proeminent. Not only that, disturbances in emotions begin escalating, some
expressed in a stronger tone, while others hardly able to be exhibited at all.

Daily tasks must be simplified to the level the person with Alzheimer's can accomplish them, and as the external attention needed rises,
place them in special care centers where experienced caretakers can easily reach out.

\subsubsection*{Late-stage (Severe)} %* -- Introduction/Symptoms/Late-stage
This final stage of AD is categorized by vital losses in the ability to function at all. Patients stop reacting to outside factors
altogether, and even initiating conversations. In due course, pain becomes impossible to verbalize, and as such, hourly check-ups
are necessary. \cite{AA1}

\subsection{Diagnosis Process} %* -- Introduction/DiagnosisProcess
AD diagnosis can only be carried out upon gathering a variety of complex data, which includes medical history, assessments of
cognitive and physical skills, neurological exams, brain scans, blood tests and cerebrospinal fluid.

Medical history consists of modifications in how the patient behaves over the course of time, past and present medical concerns
and even the undergoing medication. Besides these, information about other family members' health conditions is obtained, since,
as previously mentioned, genes do play a role in increasing the risk, or protecting against Alzheimer's Disease.

An overall health status is evaluated, involving commonly met questions about diet, blood pressure and pulse, checking
the quality of breathing and sample taking for testing.

Cognitive tests' purpose is to express a general view whether memory impairment takes a toll in the daily life of the diagnosed,
and to shed light on the awareness of the disease. A number of tests are simple - tasks of remembering sequences of words, or
mathematic operations, but there also exist those that take a longer period of time, alongside with raised levels of attention.

The neurological examination typically implies assessing the patient's nervous system, where a physician tries to distinguish
between the possibilities of the disease to be a different brain disorder instead of AD - brain tumors or Parkinson's Disease.

Brain imaging is used to form 3D and 2D, functional and structural scans of the brain, through which experts can point out
characteristics specific to Alzheimer's Disease. One such mark is the presence of higher concentrations than normal of amyloid
beta ($A\beta $ or Abeta), peptides considered the essential part of amyloid plaques. This specific part of the diagnosis
process typically serves only as a last resort. \\
\cite{AA2}

% todo: fix this shit sometimes, sounds awful

\subsection{Imaging Modalities Involved in Alzheimer's Disease}
Through recent technological advancements, brain imaging's role has shifted to a crucial one. By and large, imaging has
expanded into various different modalities, each with their own strengths and weaknesses, but combined lead to a better
analysis of AD's effects. \cite{Johnson2012BrainII}

\subsubsection*{Amyloid PET}
Amyloid Positron Emission Tomography is a non-invasive technique, which locates amyloid plaques.
Due to its unavailability and expensive price, this method has not seen increase in popularity.
\\
\cite{Chapleau2022TheRO}

\begin{figure}[htbp]
    \centering
    \includegraphics[width=0.65\textwidth]{figs/amyloid-pet.jpeg}
    \caption{Amyloid PET images}
    \label{fig:amyloid-pet}
\end{figure}

\subsubsection*{FDG PET}
Fluoro-deoxy-D-glucose (FDG) PET showcases synaptic activity, because the brain's primary energy comes from glucose.
The Fluorine part of the FDG comes as a consequence of the convenient dynamic it has with Positron Emission Tomography,
which easily detects it.

\begin{figure}[htbp]
    \centering
    \includegraphics[height=0.2\textheight,width=0.40\textwidth]{figs/fdg-pet.png}
    \caption{Transitional FDG-PET scans}
    \label{fig:fdg-pet}
\end{figure}

\subsubsection*{Structural MRI}
Structual Magnetic Resonance Imaging (sMRI) is a non-invasive method applied to observe pathology and anatomy of the brain
by emitting radiofrequency pulses in sequences. Its main purpose is to exhibit brain atrophy, associated with shrinking size
due to neuron counts declining. One of its drawbacks is that hallmarks of AD cannot be detected, and also atrophy isn't
specific to the disease discussed.

\subsubsection*{Functional MRI}
As the previous method, the functional variant of MRI is non-invasive as well, but, on the other hand, provides
scientists a neuronal activity mapping of the brain. A few of them require the patient to perform certain cognitive tasks
during the scanning process, and there are also some which need the brain to be found in a specific resting state.
Functional MRI's setback is the necessity of lack of motion, and any of the patient's movements could lead to faulty
data.


\newpage


\bibliographystyle{plainnat}
\bibliography{references}

\end{document}
