\documentclass[a4paper]{article} % Declares the doc type known as its class, controls overall appearance of the doc
% Anything here is Preamble
% Document "setup" section

\title{Bachelor Thesis - Report 1}
\author{Ichim Ștefan}
\date{19 March 2024}

\begin{document} % Marks beginning of body

\maketitle % Typesets title, author and date from preamble

\tableofcontents

\vspace{14pt}
\hrule

\section{Title}
Alzheimer's Disease Prediction

%%%%%%%%%%%%%%%%%%%%%%%%%%%%%%%%%%%%%%%%%%%%%%%%%%%%%%%%%%%%%%%%%%%%%%%%%%%%%%%%%%%%%%%%%%%%
\vspace{14pt}
\hrule
\section{Table of Contents}

\subsection{Abstract}
Summarize the future sections in this research article.

\subsection{Introduction}
Elaborate the background of the related domains which the article tackles - Deep Learning and Alzheimer's Disease.

\subsection{Related Work}
Enumerate approaches from different state-of-the-art articles in line with the topic, their advantages, disadvantages and
further improvements.

\subsection{Dataset}
Describe possible datasets to be used, their origin, availability and differences.

\subsection{Proposed Approach}
Detail the researched techniques, starting from the reasoning, then presenting them at a lower level.

Compare them to other related works.

Furthermore, explain the methods of evaluation and loss functions tested, which ones were chosen and for what reason.

\subsection{Results and Experiments}
List the results of various techniques and approaches using graphs and tables.

\subsection{Discussion}
Describe the advantages and disadvantages of the paper and its applications.

\subsection{Conclusions and future work}
A final summarization of the paper, which intends to remind the reader of the previous sections,
as well as put forward future improvements revealed throughout researching the topic.

\subsection{References}
An ordered list of articles which will have been referenced throughout this research paper.

%%%%%%%%%%%%%%%%%%%%%%%%%%%%%%%%%%%%%%%%%%%%%%%%%%%%%%%%%%%%%%%%%%%%%%%%%%%%%%%%%%%%%%%%%%%%
\vspace{14pt}
\hrule
\section{Plan of Activity}
First and foremost, before understanding the state-of-the-arts, I consider it valuable to learn the way medicine has approached
the Alzheimer's Disease: how the images are obtained and how the disease diferentiates itself from others which revolve around
brain scans.

Secondly, a thorough study of the state-of-the-arts is necessary in order to point out the way Alzheimer's Disease Prediction
has evolved in time, outline strengths and weaknesses, as well as where there could be room for improvements.

Another important step is making sure I can access datasets and compare my approach with the ones that apply on the ones I
have permission to use, since this topic implies utilizing human brain images of patients with Alzheimer's Disease.

At this point, I will have acquired enough information and reference papers to begin both writing the thesis (begin the
Abstract and finalize the Introduction, Related Work and Dataset sections.) and implementing the technical part.

Upon succeeding with implementing the models, I will begin the evaluation and experimenting part and expect failure in
the beginning and with each iteration bring slight improvements of past architectures until satisfied with the results.

Finally, after different experiments, either an improvement or display of previously not tested combinations of
architectures will result which will be presented in the later sections of the paper.

In light of the above, I hope to have achieved valuable knowledge in the intertwined domains of medicine and artificial intelligence,
while also bringing useful information to the chosen topic.

%%%%%%%%%%%%%%%%%%%%%%%%%%%%%%%%%%%%%%%%%%%%%%%%%%%%%%%%%%%%%%%%%%%%%%%%%%%%%%%%%%%%%%%%%%%%
\vspace{14pt}
\hrule
\section{Motivation}
The subjects that peaked my interest the most have been the ones surrounding Artificial Intelligence,
some aspects more than other, for example models, networks of any type (varying from Bayesian to lately Deep Neural ones).

However, I have yet to use the information in a personal interest.
Here one of my all-time biggest curiosity presents itself: the human nervous system.

Thus, my motviation comes from the will to apply techniques learned from Artificial Intelligence in the domain of medicine,
due to my long time interest in it and also to study and understand the way our nervous system functions.

\end{document} % Marks ending of body