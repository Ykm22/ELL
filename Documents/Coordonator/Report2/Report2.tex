\documentclass[a4paper]{article}

\usepackage{hyphenat}

\title{
    Bachelor Thesis - Report 2
}
\author{Ichim Ștefan}
\date{2 April 2024}

\begin{document}

\maketitle

\newpage

\section*{Bibliographical Documentation}
\begin{enumerate}
      \item Alsubaie, M.G.; Luo, S.; Shaukat, K. Alzheimer's Disease Detection Using Deep Learning on Neuroimaging:
            A Systematic Review. Mach. Learn. Knowl. Extr. 2024, 6, 464-505. https://doi.org/10.3390/make6010024
            \begin{itemize}
                  \item This paper describes detailed explanations on notions from medicine used in this study,
                        the state-of-the-arts approaches to Alzheimer's Disease Detection, the datasets and the multi-modal
                        input methods
                  \item Highly valuable research paper, giving important initial directions in the chosen topic
                  \item Its place is to make an overhaul on the 2018-2023 improvements in AD detection
            \end{itemize}
      \item Matthews PM, Jezzard P. Functional magnetic resonance imaging. J Neurol Neurosurg Psychiatry.
            2004 Jan;75(1):6-12. PMID: 14707297; PMCID: PMC1757457. https://pubmed.ncbi.nlm.nih.gov/14707297/
            \begin{itemize}
                  \item Matthews and Jezzard explain one of the most commonly used types of fMRI, the
                        blood oxygenation level dependent (BOLD) functional MRI, and the advantages
                        it brings versus Electroencephalography (EEG) scans
                  \item Less useful paper, nevertheless I considered it important to grasp the way MRI data
                        is gathered for a better understanding of the problem.
                  \item Helped understanding the importance of relaxation times of the brain
                        $T_1$, $T_2$, $T_2^*$
            \end{itemize}
      \item Gorgolewski KJ, Auer T, Calhoun VD, Craddock RC, Das S, Duff EP, Flandin G, Ghosh SS, Glatard T,
            Halchenko YO, Handwerker DA, Hanke M, Keator D, Li X, Michael Z, Maumet C, Nichols BN, Nichols TE, Pellman J,
            Poline JB, Rokem A, Schaefer G, Sochat V, Triplett W, Turner JA, Varoquaux G, Poldrack RA. The brain imaging
            data structure, a format for organizing and describing outputs of neuroimaging experiments. Sci Data. 2016 Jun 21;
            3:160044. doi: 10.1038/sdata.2016.44. PMID: 27326542; PMCID: PMC4978148. \\
            https://pubmed.ncbi.nlm.nih.gov/27326542/
            \begin{itemize}
                  \item Paper sheds light on the importance of a standardization of how data regarding neuroimaging is stored,
                        specifying the previous standards, among which some weren't adopted (XML) and some got selected forward (JSON, NIfTi, tsv)
                  \item Relatively important due to the need of understanding how datasets work in other scientific fields for an easier
                        time utilizing them
                  \item This paper stands at a base level in the economy of the research, laying out the foundations which data gatherers
                        abide to
            \end{itemize}
            \newpage
      \item Jack CR Jr, Bernstein MA, Fox NC, Thompson P, Alexander G, Harvey D, Borowski B, Britson PJ, L Whitwell J, Ward C,
            Dale AM, Felmlee JP, Gunter JL, Hill DL, Killiany R, Schuff N, Fox-Bosetti S, Lin C, Studholme C, DeCarli CS, Krueger G,
            Ward HA, Metzger GJ, Scott KT, Mallozzi R, Blezek D, Levy J, Debbins JP, Fleisher AS, Albert M, Green R, Bartzokis G,
            Glover G, Mugler J, Weiner MW. The Alzheimer's Disease Neuroimaging Initiative (ADNI): MRI methods. J Magn Reson Imaging.
            2008 Apr;27(4):685-91. doi: 10.1002/jmri.21049. PMID: 18302232; PMCID: PMC2544629. https://doi.org/10.1002/jmri.21049
            \begin{itemize}
                  \item Article describes the methods of how the ADNI dataset was created, which type of scans were given more attention to,
                        emphasizing the importance of maximizing scientific value while minimizing patient burden.
                  \item I had received permission to use this dataset, but had found many notations given without explanation. Having read
                        through this paper many of the questions were answered, thus I consider this paper of high importance.
                  \item It is yet another fundamental article in the topic of research, describing the scanning process and offering
                        to the scientific public a mass of data for researching Alzheimer's Disease
            \end{itemize}
      \item Klein A, Andersson J, Ardekani BA, Ashburner J, Avants B, Chiang MC, Christensen GE, Collins DL, Gee J, Hellier P, Song JH,
            Jenkinson M, Lepage C, Rueckert D, Thompson P, Vercauteren T, Woods RP, Mann JJ, Parsey RV. Evaluation of 14 nonlinear deformation
            algorithms applied to human brain MRI registration. Neuroimage. 2009 Jul 1;46(3):786-802. doi: 10.1016/j.neuroimage.2008.12.037.
            Epub 2009 Jan 13. PMID: 19195496; PMCID: PMC2747506. \\
            https://pubmed.ncbi.nlm.nih.gov/19195496/
            \begin{itemize}
                  \item This paper ranks algorithms used for one of the most important pre-processing steps, image registration,
                        which represents the positioning of the brain scans of different patients in the same positions with respect to others,
                        to lead to easier time handling the image collections
                  \item The use this paper brings me is to understand data pre-processing steps from a different area of expertise
                  \item But with that being said, its importance compared to other papers could be considered lower
            \end{itemize}
      \item Jenkinson M, Bannister P, Brady M, Smith S. Improved optimization for the robust and accurate linear registration and motion correction
            of brain images. Neuroimage. 2002 Oct;17(2):825-41. doi: 10.1016/s1053-8119(02)91132-8. PMID: 12377157. \\
            https://pubmed.ncbi.nlm.nih.gov/12377157/
            \begin{itemize}
                  \item This publication describes a new approach to previous algorithms used in brain scans pre-processing, such as image registration
                        and motion correction, which were shown to lead to a local minimum
                  \item This new approach, simply put, expands the view at a step in the search algorithm, which, while increasing
                        computation, can have the chance of escaping a local minimum. That being said, there is no guarantee in finding
                        the global minimum, similarly to any global minimum algorithm without infinite time
                  \item Main use to me was the explanation of another pre-processing step, motion correction, which represents solving movements
                        of either the clinical patients or the scanner, due to the fact that the images are taking at intervals of time, and a
                        certain correlation between consecutive scans must be kept
            \end{itemize}
      \item Tom Fawcett, An introduction to ROC analysis, Pattern Recognition Letters, Volume 27, Issue 8, 2006, Pages 861-874,
            ISSN 0167-8655, \\
            https://doi.org/10.1016/j.patrec.2005.10.010.
            \begin{itemize}
                  \item This piece of research presents ROC graphs as an evaluation metric and their advantages over other commonly
                        used metrics in classification problems, such as accuracy, sensitivity, specificty.
                  \item I would consider this article moderately important, having noticed in diagnostic classification that this
                        metric is particularly useful, in the way the trade-off between benefits and costs has agency over an
                        algorithm in this topic (benefits - True Positives, costs - False Positives).
                  \item The paper stands at an introductory level in the topic of research, given the fact it summarizes the importance
                        of an evaluation metric, previously unused by me.
            \end{itemize}
      \item Bae JB, Lee S, Jung W, Park S, Kim W, Oh H, Han JW, Kim GE, Kim JS, Kim JH, Kim KW. Identification of Alzheimer's disease
            using a convolutional neural network model based on T1-weighted magnetic resonance imaging. Sci Rep. 2020 Dec 17;10(1):22252.
            doi: 10.1038/s41598-020-79243-9. PMID: 33335244; PMCID: PMC7746752. \\
            https://pubmed.ncbi.nlm.nih.gov/33335244/
            \begin{itemize}
                  \item This article summarizes a Deep Learning approach to the classification, opting for 2D image inputs and a binary classification
                        between Cognitive Normal(CN) and Alzheimer's Disease(AD) utilizing a CNN-like architecture.
                  \item Highly important article, underlines the importance of the datasets, specifically the demographic origin and education level,
                        explaining that, for example, Caucasians with higher education level(from ADNI dataset) will have a different brain shape
                        compared to Koreans or Japanese of lower or medium education level.
            \end{itemize}
      \item Minh Nguyen, Tong He, Lijun An, Daniel C. Alexander, Jiashi Feng, B.T. Thomas Yeo, Predicting Alzheimer's disease progression using deep
            recurrent neural networks,NeuroImage, Volume 222, 2020, 117203, ISSN 1053-8119, https://doi.org/10.1016/j.neuroimage.2020.117203.
            \begin{itemize}
                  \item Compared to paper 8, this article utilizes a different network architecture, an RNN, specifically a minimalRNN,
                        its predecessor being an LSTM which had problems with overfitting. Switching over to an RNN, the problem had gotten fixed.
                        Their goal was to predict at indefinite amount of time points in the future the presence of Alzheimer's Disease.
                  \item Another point this paper brings up is how missing data is treated, suggesting that it would be wiser not to ignore it, but
                        to fill it up, and they end up describing 3 methods of doing so, leading to better experiment reults.
                  \item All in all, this article summarizes a contest awarded approach, specifically The Alzheimer's Disease Prediction
                        Of Longitudinal Evolution(TADPOLE), where it had achieved 2nd place in 2020.
            \end{itemize}
\end{enumerate}

\section*{Acknowledgement}

This work is the result of my own activity, and I confirm I have neither given, nor received unauthorized assistance for this work.

I declare that I did not use generative AI or automated tools in the creation of content or drafting of this document.

\end{document}